\chapter{Teoretický základ}

\section{Popis mikrořadiče AVR ATmega328}

Mikrořadič ATmega328 od firmy AVR je jeden z řady procesorů používaných v Arduinu.  Mikrořadič má 32kB paměti programu EPROM, 2kB paměti RAM.
...
Mikrořadič je dále vybaven řadou periferních obvodů integrovaných přímo na čipu.  Jedná se zejména o čítače/časovače ...doplnit..., obvody sériové komunikace UART USART, SPI, \IIC, ...doplnit...

%TODO: pro I2C zavést makro \I2C => \textsuperscript{2}C

\IIC

\begin{itemize}
\item UART, USART
\item SPI
\item \IIC
\item DAC -- digitálně analoový převodník
\end{itemize}

\subsection{Rozhraní SPI}

Rozhraní SPI je \doplnit{SPI} používá se sériové komunikaci s obvody stejného rozhraní.  V našem případě je to komunikace s obvody ETH rozhraní \doplnit{název ETH obvodů}.

\subsection{Rozhraní I2C}

\subsection{DAC - digitálně analogový převodník}

Tento obvod převádí analogové

\section{Popis obvoud ETH rozhraní Wizz}

Pro internetovou komunikaci je možno použít několik obvodů.  Pro Arduino jsou dostupná dvě řešení.  Jedno řešení je s čipem ...doplnit... od firmy Microchip a druhé s čipem Wiz...doplnit.. od firmy ...doplnit...

\section{Internetová komunikace}

\subsection{TCP}
\TBD{Popsat základ protokolu TCP.}

\subsection{HTTP}

\TBD{Popsat základ HTTP komunikace.}


%%EOF
